\documentclass{article}
\usepackage{graphicx} % Required for inserting images

\usepackage[utf8]{inputenc}   % dla UTF-8
\usepackage[T1]{fontenc}      % obsługa polskich liter
\usepackage[polish]{babel}

\title{Metody Programowania Blok 3}
\author{Krzysztof Peszko }
\date{April 2025}

\begin{document}

\maketitle

\section{Zadanie 1 (Zestaw 7)}
Główną ideą rozwiązania, jest puszczenie BFS z wsyztskich czarnych wierzchołków, aż jakieś BFS-y różnych wierzchołków się spotkają.
\subsection{ Przygotowanie}
Dla każdego wierzchołka ustawamy 3 tablice:
\\
-odl[] - odległość od najbliżeszgo czarnego wierzchołka
\\
-ost[] - element, z którego aktywowana ten wierzchołek(więcej póżniej w ramach algorytmu BFS.
\\
-leader[] - czarny element z którego zaczyna się ścieżka
\subsection{Równoległy BFS}
Robimy klasyczną interpretacje BFS-a, tylko zmieniając kilka rzeczy \\
Na start, wrzucamy nie jeden wierzchołek, a tyle wierzchołków ile jest czarnych wierzchołków. \\
W ramach jednego wywołania, trzymamy krawędź, która dodajemy (z wierzchołka u, który już obsłużyliśmy do wierzchołka v) \\
W przypadku startowym wrzucamy, że poprzedni wierzchołek to wirtualny wiezrchołek 0.
\subsection{Algorytm obsługi kolejnych krawędzi}
Krawędz ma dwie wartości u, czyli wierzchołek, z którego przyszła i v czyli wiezchołek do któejo doszła.
\subsubsection{Pierwsza krawedź ($u == 0$)}
Jeśli krawędź jest pierwsza, to ustawiamy: \\
$odl[v] = 0$ \\
$ost[v] = -1$ \\
$leader[v] = v$ \\
Oraz na koniec aktywujemy wierzchołek v.
\subsubsection{Co jeśli wierzchołek v nie był aktywowany wcześniej}
Wówczas ustwawiamy wiezrzchołkowi $v$ wartości: \\
$odl[v] = odl[u] + 1$ \\
$ost[v] = u$ \\
$leader[v] = leader[u]$ \\
Po ustawieniu aktywujemy wierzchołek $v$
\subsubsection{Co jeśli wierzchołek v był aktywowany wcześniej}
Wówczas sprawdzmy rzecz, czy $leader[v] = leader[u]$. Jeśli odpowiedż jest poporawna, oznacza, to, że znaleźliśmy ścieżkę, prowadzącą z wierzchoka $w$ do wierzchołka $w$, co nas nie interesuje, więc taki przypadek po prostu ignorujemy.\\
Jeśli chodzi, o negatywną odpowiedź to wóczas oznacza, że znaleźliśmy ścieżkę między dwoma różnymi wierzchołkami. Długość tej ścieżki obliczamy jako $l = odl[u] + odl[v] + 1$, w miarę prosty wzorek, $odl[u]$ od odległość między czarnym wierzchołkiem, a $u$, a +1 to krawędź między $u$  i $v$. \\
Niestety ścieżka, którą otrzymaliśmy nie musi być najkrótsza (może być o 1 dłuższa od najkrótszej). Dla tego musimy nie kończyć, po tym jak znajdziemy pierwszą ścieżkę. Dla ułatwienia powiem, że kończymy, cały BFS(da się lepiej, ale nie zmienia to złożoności pesymisteczej) i bierzemy minimumu, bo długości wszystkich ścieżek. 

\subsubsection{Złożoność Algorytmiczna}
Każdy wierzchołek aktywujemy raz, a po każdej krawędzi przechodzimy maksymalnie 2 razy, więc złożoność to $O(n+m)$.

\subsubsection{Poprawność}
Łatwo da się odowodnić (przez sprzeczność), że najkrótsza ścieżka zostanie wybrana. Natomiast reszta jest równoważna z działaniem algorytmu BFS.

\subsection{Jak ze znalezionej najkrótszej ścieżki odzyskać, kolejny elementy na ścieżce}
Wystarczy nam zapamiętać tylko 3 informacje długość, i wierzchołki ($u$ i $v$), gdzie doszło do spotkania.\\
Tworzymy więc tablice kolejnych wierzchołków na ścieżce, a korzystamy z tego przy użyciu tablicy $ost[]$. Działa to tak, żę zaczynamy od $u$ i a potem tworzymy tymczasową wartość $t = ost[u]$ i wrzucamy to na początek listy. Kontynujemy to, wrzucając w pętli $while$, kolejne elementyy, które definiujemy, jako: $t = ost[t]$ i robimy to aż $ost[t]$ jest równy $-1$. Robimy podobmnie z wierzchołkami generowanymi w ten sposób przez $v$, tylko, że zamiast na początek, wrzcuamy na koniec listy. Można zauważyć, że powstała lista zaczyan się od 1 czarnego wierzchołka przechodzi po sąsaidach, dochodzi do $u$, a następnie jest $v$ i potem idziemy nadal po sąsiadach, aż dochodzimy do czarnego wierzchołka. Czyli otrzymaliśmu kolejne numery na najkrótszej ścieżce, jakich szukaliśmy. \\
Łatwo zauważyć, że ta cześć algorytmu jest liniowa względem liczby wiezchołków, więc ma $O(n)$.
\subsection{Końcowa złożoność}
\[
    O(n+m) + O(n) = O(n+m)
\]
\end{document}
